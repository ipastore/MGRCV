\documentclass[a4paper,10pt]{article}
\usepackage{amsmath}
\usepackage{geometry}
\usepackage{graphicx}
\usepackage{array}
\usepackage{amssymb}

\geometry{margin=1in}
\setlength{\parskip}{1.5ex}
\setlength{\parindent}{0pt}

\title{Comprehensive Guide to 2D-3D Geometry and Camera Projection}
\author{David Padilla Orenga, Nacho Pastore}
\date{September 2024}

\begin{document}
\maketitle

\section*{1. Camera Projection Model and Projection Matrix}

The camera projection model describes how a 3D point in the world is projected onto a 2D image plane. This is done via the \textbf{projection matrix} $P$.

\subsection*{Projection Equation}
To project a 3D world point $\mathbf{X} = (X, Y, Z, 1)^\top$ to a 2D image point $\mathbf{x} = (x, y, 1)^\top$, we use:
\[
\mathbf{x} = P \mathbf{X}, \quad P \in \mathbb{R}^{3 \times 4}
\]
Where $P$ is the \textbf{projection matrix} and is formed by the camera's intrinsic and extrinsic parameters:
\[
P = K [R | t]
\]
Where:
\begin{itemize}
    \item $K$: \textbf{Intrinsic matrix} ($3 \times 3$), defines camera-specific parameters such as focal length.
    \item $R$: \textbf{Rotation matrix} ($3 \times 3$), describes the camera's orientation in the world.
    \item $t$: \textbf{Translation vector} ($3 \times 1$), describes the camera's position in the world.
\end{itemize}

\subsection*{Intrinsic Matrix Example}
\[
K = \begin{bmatrix} f_x & 0 & c_x \\ 0 & f_y & c_y \\ 0 & 0 & 1 \end{bmatrix}
\]
Where $f_x, f_y$ are the focal lengths and $c_x, c_y$ are the coordinates of the principal point.

\textbf{Tip: Projection Matrix Construction} \\
Given a transformation matrix $T_{w_{c1}}$ ($4 \times 4$) for camera 1’s pose:
\[
T_{w_{c1}} = \begin{bmatrix} R & t \\ 0 & 1 \end{bmatrix}, \quad R = \text{rotation}, \, t = \text{translation}
\]
The inverse $T_{c1_w}$ for the world-to-camera transformation is:
\[
T_{c1_w} = T_{w_{c1}}^{-1} = \begin{bmatrix} R^\top & -R^\top t \\ 0 & 1 \end{bmatrix}
\]
This technique is essential for changing coordinate frames.

\section*{2. Epipolar Geometry and Fundamental Matrix}

Epipolar geometry defines the geometric relationship between two camera views of the same scene. The \textbf{fundamental matrix} $F$ describes this relationship between corresponding points in two images.

\subsection*{Fundamental Matrix $F$}
The fundamental matrix satisfies the \textbf{epipolar constraint}:
\[
\mathbf{x'}^\top F \mathbf{x} = 0
\]
Where:
\begin{itemize}
    \item $\mathbf{x} = (x, y, 1)^\top$: Point in the first image.
    \item $\mathbf{x'} = (x', y', 1)^\top$: Corresponding point in the second image.
    \item $F$: Fundamental matrix ($3 \times 3$).
\end{itemize}

\subsection*{8-Point Algorithm for $F$}
The \textbf{8-point algorithm} estimates $F$ given at least 8 pairs of corresponding points:
\[
A \mathbf{f} = 0, \quad A \in \mathbb{R}^{N \times 9}, \quad \mathbf{f} = \text{flatten}(F)
\]
Solve this using SVD, and then enforce the \textbf{rank-2 constraint}:
\[
U, S, Vt = \text{SVD}(F), \quad S[-1] = 0, \quad F = U \cdot S \cdot Vt
\]

\subsection*{Epipolar Line}
Given a point $\mathbf{x}$ in the first image, the corresponding epipolar line $l_2$ in the second image is computed as:
\[
l_2 = F \mathbf{x}
\]
The epipolar line $l_2 = [a, b, c]^\top$ represents the line equation $ax + by + c = 0$. To plot the line, use the image's width to find the $x$ coordinates and solve for the corresponding $y$ values:
\[
y = -\frac{a x + c}{b}
\]

\textbf{Tip: Computing the Epipole}
The \textbf{epipole} is the point where all epipolar lines intersect. It lies in the null space of $F$. Compute the epipoles as:
\[
e_2 = \text{null space}(F), \quad e_1 = \text{null space}(F^\top)
\]
Use SVD:
\[
U, S, Vt = \text{SVD}(F), \quad e_2 = Vt[-1], \quad e_1 = U[:, -1]
\]
Normalize to homogeneous coordinates.

\section*{3. Essential Matrix and Decomposition}

The \textbf{essential matrix} is a calibrated version of the fundamental matrix that relates points between two cameras with known intrinsic parameters:
\[
E = K_2^\top F K_1
\]
Where $K_1$ and $K_2$ are the intrinsic matrices of the two cameras.

\subsection*{Decomposing $E$ into Rotation and Translation}
The essential matrix can be decomposed into \textbf{rotation} and \textbf{translation} using SVD:
\[
E = U \Sigma V^\top
\]
From this, we extract two possible rotations $R_1, R_2$ and two possible translations $t, -t$:
\[
R_1 = U W V^\top, \quad R_2 = U W^\top V^\top, \quad t = U[:, 2]
\]
Where:
\[
W = \begin{bmatrix} 0 & -1 & 0 \\ 1 & 0 & 0 \\ 0 & 0 & 1 \end{bmatrix}
\]
This gives four possible camera poses. To find the correct pose, we check which configuration places the triangulated points in front of both cameras (i.e., with positive depth).

\textbf{Tip: Recovering Pose from $E$} \\
Use the essential matrix $E$ to recover the relative pose (rotation $R$ and translation $t$) between two cameras. You can decompose $E$ to obtain two possible solutions for each. Use triangulation to validate which pose is correct.

\section*{4. Triangulation of 3D Points}

Triangulation allows us to recover the 3D position of a point given its 2D projections in two images and the corresponding projection matrices $P_1$ and $P_2$.

\subsection*{Triangulation Algorithm}
Given two projection matrices $P_1$ and $P_2$ and corresponding points $(\mathbf{x}_1, \mathbf{x}_2)$ in two images:
\[
A \mathbf{X} = 0
\]
Where:
\[
A = \begin{bmatrix} x_1 P_3^1 - P_1^1 \\ y_1 P_3^1 - P_2^1 \\ x_2 P_3^2 - P_1^2 \\ y_2 P_3^2 - P_2^2 \end{bmatrix}, \quad \mathbf{X} = (X, Y, Z, 1)^\top
\]
Solve for $\mathbf{X}$ using SVD:
\[
\mathbf{X} = Vt[-1], \quad \mathbf{X} /= \mathbf{X}[-1]
\]
The result $\mathbf{X}$ gives the 3D coordinates in homogeneous form.

\subsection*{Projection of 3D Points}
To project a 3D point $\mathbf{X}$ onto an image:
\[
\mathbf{x} = P \mathbf{X}
\]
Where $P$ is the projection matrix and $\mathbf{X}$ is the 3D point in homogeneous coordinates.

\textbf{Tip: Using the Projection Matrix} \\
For triangulation using two images:
\begin{itemize}
    \item Compute the projection matrices $P_1 = K_1 [R_1 | t_1]$ and $P_2 = K_2 [R_2 | t_2]$.
    \item Solve $A \mathbf{X} = 0$ to get the 3D coordinates $\mathbf{X}$ of the point.
\end{itemize}

\section*{5. Direct Linear Transform (DLT)}

The \textbf{Direct Linear Transform (DLT)} algorithm estimates the homography matrix $H_{21}$ that maps points from one image to another. Given $N$ corresponding points $(\mathbf{x}_1, \mathbf{x}_2)$ between two images:
\[
A \mathbf{h} = 0, \quad A \in \mathbb{R}^{2N \times 9}, \quad \mathbf{h} = \text{flatten}(H)
\]
Construct $A$ from point correspondences, and solve using SVD:
\[
H = Vt[-1].reshape(3, 3), \quad H /= H[2, 2]
\]

\subsection*{Inverse and Transpose of Homography}
To find the inverse homography $H_{12}$ (mapping points back to the first image), compute the inverse:
\[
H_{12} = H_{21}^{-1}
\]
You \textbf{cannot} simply transpose the homography matrix $H$ to invert it because it is not an orthogonal matrix.

\section*{6. RMSE for Homography Estimation}

Once the homography matrix $H_{21}$ is estimated, the RMSE (Root Mean Square Error) between the projected points and ground-truth points is calculated to evaluate accuracy.

\subsection*{RMSE Formula for Homography}
To compute RMSE for homography:
\[
\text{RMSE} = \sqrt{\frac{1}{N} \sum_{i=1}^N ||\mathbf{x}'_i - \hat{\mathbf{x}}_i||^2}
\]
Where:
\begin{itemize}
    \item $\mathbf{x}'_i$: Actual point in the second image.
    \item $\hat{\mathbf{x}}_i = H_{21} \mathbf{x}_i$: Projected point using homography.
    \item $N$: Number of point correspondences.
\end{itemize}

\section*{7. Summary of Key Equations and Tips}

\begin{itemize}
    \item \textbf{Projection Equation}: $\mathbf{x} = P \mathbf{X}$.
    \item \textbf{Fundamental Matrix Constraint}: $\mathbf{x'}^\top F \mathbf{x} = 0$.
    \item \textbf{Essential Matrix from $F$}: $E = K_2^\top F K_1$.
    \item \textbf{Triangulation Equation}: $A \mathbf{X} = 0$, where $A$ is formed from projection matrices and image points.
    \item \textbf{Homography Estimation}: $H_{21}$ from $A \mathbf{h} = 0$ using DLT.
    \item \textbf{Decomposing $E$}: $E = U \Sigma V^\top$ to get $R_1, R_2, t$.
\end{itemize}

\end{document}