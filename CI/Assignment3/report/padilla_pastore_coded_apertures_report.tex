\documentclass[a4paper,10pt]{article}
\usepackage{graphicx}
\usepackage{amsmath}
\usepackage{hyperref}
\usepackage{caption}
\usepackage{subcaption}
\usepackage[margin=1in]{geometry}

\title{Assignment3: High-Dynamic-Range-Imaging}
\author{David Padilla Orenga \\ Ignacio Pastore Benaim}
\date{\today}

\begin{document}
\maketitle
\thispagestyle{empty}

\section{Results and Discussion}
Figures~\ref{fig:responseFunctions} and~\ref{fig:radianceMaps} show the response functions and the resulting radiance maps for the different combinations of weighting and smoothing:
\begin{itemize}
    \item \textbf{Tent Weighting + \(\lambda=50\):} Produces a smooth, stable response curve.  Saturated or dark pixels are de-emphasized, leading to fewer artifacts in the HDR.
    \item \textbf{Tent Weighting + \(\lambda\approx0\):} With almost no smoothness constraint, the response function can become more jagged, though mid-range pixel mapping is still emphasized.
    \item \textbf{No Weighting + \(\lambda=50\):} Gives a smooth CRF but lacks emphasis on mid-range pixels, so extremely dark or bright values can introduce noise or artifacts in the HDR.
    \item \textbf{No Weighting + \(\lambda\approx0\):} Tends to be noisier overall, and bright or dark regions can dominate.  The CRF is less stable.
\end{itemize}

\paragraph{Observations.}
\begin{itemize}
    \item \textbf{Influence of \(\lambda\):} A larger \(\lambda\) yields smoother CRFs.  In practice, \(\lambda=50\) gave visually pleasing results, whereas \(\lambda \approx 10^{-6}\) sometimes overfits noise.
    \item \textbf{Influence of Weighting:} Tent weighting avoids saturations at the extremes by de-emphasizing high and low pixel values, typically producing better final HDR images.  Constant weighting can result in more noise in the darkest or brightest regions.
    \item \textbf{Preferred Images:} We generally preferred the results from \textit{Tent Weighting with moderate smoothing}, as it offered a stable response curve and balanced radiance maps.
\end{itemize}

\section{Conclusion}
In this part of the lab, we demonstrated how different parameters (\(\lambda\) and 
weighting functions) affect the camera response function and final HDR radiance map.  
he tent weighting function with a moderate smoothing parameter (\(\lambda=50\)) produced the most 
stable and visually appealing results.  In subsequent parts, we will use these HDR images for global 
and local tone mapping experiments.


\section{ANOTHER Conclusion}

Explanation of Differences: The three response functions (CRFs) 
display different characteristics based on the smoothing and weighting methods used: 1. 
Tent Weighting + \(\lambda = 50\) (Full Method: - Tent weighting assigns more weight to pixel
alues at the middle of the exposure range and less weight to the extremely bright and dark pixel values, 
hich reduces the influence of noise from those extreme values in darker/shadow or overexposed/highlight areas. 
The response function achieved smoother transitions between the pixel values, with fewer artifacts in the
HDR resulting from outliers or noisy pixels. The response curve appears relatively stable and gradually 
changes, especially near mid-range exposures. 2. Tent Weighting + \(\lambda \approx 0\) 
(No Smoothness: - With such a low \(\lambda\), the smoothing term is practically eliminated, 
causing the CRF to be much more jagged. It might capture finer details, but at the cost of noise 
and irregularities. - The lack of smoothing increases the visibility of abrupt changes in response
function between similar exposure values, which is not ideal for HDR reconstruction, as it leads 
to inconsistencies in the final image. 3. No Weighting + \(\lambda = 50\) 
(Full Method, No Weighting:- The absence of the weighting function means that bright and dark pixels contribute as much as those 
in the middle ranges. This causes the final response function to be highly influenced by extreme 
pixel values, which are typically outliers or noise—in the darkest regions or highlights, which 
might reduce the robustness of the reconstruction. The image is more prone to artifacts, especially
when these extreme values "overpower" the middle ones. 4. No Weighting + \(\lambda \approx 0\) 
(No Smoothness, No Weighting - Without both the smoothing term and the weighting function, 
this results in an unstable, noisy CRF. This model does not emphasize middle-range pixels and allows
 noise or clipping in both the brightest and darkest areas to more heavily affect the response function. 
 The resulting CRF is irregular and jagged, reflecting unwanted noise and artifacts from the data. 
Mixed Paragraph with Results, Discussion, Observations, and Conclusions (without figures): 
 In this part of the HDR imaging process, we have investigated the effects of different weighting 
 schemes and smoothing parameters on the recovery of the camera response function (CRF) and the resulting
  HDR radiance maps. We found that the response function with tent weighting and moderate smoothing 
  (\(\lambda = 50\)) produced a stable and visually pleasing result. The tent weighting function effectively
   emphasized mid-range pixel values, reducing influence from saturated or underexposed pixels, and 
   smoothing the outcome. On the other hand, using a very low smoothing parameter (\(\lambda \approx 0\)) 
   resulted in a jagged CRF, capturing more noise and less accurately mapping the exposure values, which 
   negatively affected the quality of the HDR image. Additionally, applying no weighting function led to 
   more artifacts, especially at extreme pixel values, making the radiance map appear noisy in very 
   bright or dark regions. The results showed that combining tent weighting with moderate smoothing
   (\(\lambda = 50\)) produces the most visually appealing and stable response functions, leading to
    HDR images with better shadow and highlight balance. Therefore, for most applications, 
    favoring tent weighting with moderate smoothing is ideal, as it ensures a robust reconstruction 
    with minimal distortion and effective exposure handling

When we use tent weighting with a moderate smoothing parameter, 
the resulting response function is smooth and avoids being dominated 
by under- or over-exposed pixels; this yields a stable mapping from pixel values 
to log exposure across most of the brightness range. With the same tent weighting but negligible
 smoothing, the curve can become more jagged, reflecting local variations or noise in the data—especially 
 near the midrange, where the tent weighting gives those pixels more importance. If we remove the weighting 
 but keep smoothing, the function remains relatively smooth, yet high and low pixel values influence the shape 
 more strongly, sometimes resulting in a steeper slope at the extremes. Finally, removing both weighting and 
 smoothing can lead to an even noisier or less stable function, since extreme pixel values are weighted
  equally with midrange values and there is no regularization term to enforce smoothness. Overall, 
  these variations illustrate how the weighting scheme and smoothing parameter jointly control the 
  balance between fitting the data closely (potentially amplifying noise) and producing a stable,
  continuous response function that robustly handles saturated and very dark pixels.

\end{document}